\documentclass[12pt,spanish]{article}
\usepackage[spanish]{babel}
\selectlanguage{spanish}
\usepackage[utf8]{inputenc}
\usepackage{hyperref}
\title{{\sc Entendiendo a DESI}\\{\small\sc Reporte STAI 2017-20}}
\author{Jaime E. Forero Romero\\Departamento de F\'isica\\Universida
  de los Andes}
\begin{document}
\maketitle
\begin{abstract}
Este documento es el reporte del Semestre de Trabajo Acad\'emico
Independiente (STAI) hecho durante el per\'iodo 2017-20. 
En ese per\'iodo planeo pasar cuatro meses en el Lawrence Berkeley
National Laboratory contribuyendo al Dark Energy Spectroscopic
Instrument, un experimento de \'ultima generaci\'on dise\~nado para
estudiar la expansi\'on acelerada del Universo.
\end{abstract}

\section{Objetivos}

El objetivo principal del STAI en la propuesta que fu\'e aprobada era 
\begin{quote}
Contribuir a la colaboraci\'on internacional Dark Energy Spectroscopic
Instrument (DESI) durante una visita de cuatro meses (per\'iodo
2017-20) al Lawrence Berkeley National Laboratory en California.  
\end{quote}

Este objetivos se cumpli\'o como se ver\'a a continuaci\'on con los
resultados obtenidos.
El \'unico cambi\'o que se hizo por razones de financiaci\'on es que
la visita de cuatro meses a Berkeley Lab se dividi\'o en tres
institutos diferentes parte del Dark Energy Spectroscopic Instrument. 


Los institutos parte de DESI que se visitaron durante el STAI fueron:
\begin{itemize}
\item{Institute for Computational Cosmology de la Universidad de
  Durham en el Reino Unido.}
\item{Korea Astronomy and Space Science Institute en Corea del Sur.}
\item{Lawrence Berkeley National Laboratory en California.}
\end{itemize}


\section*{Resultados Obtenidos}

Los resultados esperados en la propuesta eran los siguientes
\begin{enumerate}
\item Diagn\'ostico de \textit{performance} de DESI a partir de los
  datos de la simulaci\'on completa del instrumento.  
\item Contribuir software para simulaci\'on de DESI en repositorios de
  acceso p\'ublico a la comunidad internacional.
\item Escribir una nota t\'ecnica interna a la colaboraci\'on DESI
  sobre los resultados principales del ejercicio de simulaci\'on.
\item Escribir y someter al menos un art\'iculo (a una revista ISI
  de primer cuartil) con resultados de este ejercicio de
  simulaci\'on que sean de inter\'es para la comunidad internacional.  
\item Solicitar una recomendaci\'on para ser   considerado como
  \texttt{Builder}  dentro de la colaboraci\'on DESI.  
 \end{enumerate}

El estado despu\'es del STAI de cada uno de los puntos anteriores es
el siguiente.

\begin{enumerate}
\item Las simulaciones se realizaron. Adjunto el pdf de la p\'agina
  interna de la colaboraci\'on que habla de los resultados de esta simulaci\'on:\\
  \verb"100% Quicksurvey End-to-End Simulation".
\item La contribuci\'on al software ha sido continua. Todas las
  contribuciones pueden ser vistas en el repositorio p\'ublico del
  proyecto:\\ \url{https://github.com/desihub}
\item En lugar de una nota t\'ecnica se hizo una presentaci\'on en el
  DESI Collaboration Meeting en Stanford
  \url{https://kipac.stanford.edu/events/desi-collaboration-meeting}. 
\item El art\'iculo sigue en preparaci\'on. En el siguiente enlace se
  puede econtrar el manuscrito que est\'a siendo preparado por la
  colaboraci\'on:\\ \url{https://www.overleaf.com/5923731xwddpv#/19875270/}
\item La solicitud fue hecha y la respuesta sobre el avance hecho en
  trabajo a la colaboraci\'on es positiva. Se adjunta la carta de
  respuesta que toma en cuenta el trabajo hecho hasta el 2016. 
A mediados del 2018 se har\'a una nueva solicitud que
  toma en cuenta el trabajo hecho en el 2017.
\end{enumerate}


\section*{Resultados Adicionales}

Durante el semestre del STAI dos publicaciones en revistas Q1 fueron aceptadas. Aparecieron
  publicadas a comienzos del 2018:

\begin{itemize}
\item{{\it Modelling the gas kinematics of an atypical Ly $\alpha$ emitting
    compact dwarf galaxy}, Jaime E. Forero-Romero  Max Gronke  Maria
  Camila Remolina-Gutiérrez Nicolás Garavito-Camargo  Mark Dijkstra.
  Publicado en Monthly Notices of the Royal Astronomical Society,
  Volume 474, Issue 1, 11 February 2018, Pages 12–19,
  \url{https://doi.org/10.1093/mnras/stx2699}} 
\item{{\it Tracing the cosmic web}, Noam I. Libeskind  Rien van de
  Weygaert  Marius Cautun  Bridget Falck Elmo Tempel  Tom Abel  Mehmet
  Alpaslan  Miguel A. Aragón-Calvo Jaime E. Forero-Romero  Roberto
  Gonzalez  Stefan Gottlöber  Oliver Hahn Wojciech A. Hellwing  Yehuda
  Hoffman  Bernard J. T. Jones  Francisco Kitaura Alexander Knebe
  Serena Manti  Mark Neyrinck  Sebastián E. Nuza Nelson Padilla  Erwin
  Platen  Nesar Ramachandra  Aaron Robotham  Enn Saar Sergei Shandarin
  Matthias Steinmetz  Radu S. Stoica  Thierry Sousbie Gustavo Yepes. 
Publicado en Monthly Notices of the Royal Astronomical Society, Volume
473, Issue 1, 1 January 2018, Pages 1195–1217,
\url{https://doi.org/10.1093/mnras/stx1976}}
\end{itemize}

Al finalizar el STAI asist\'i a al congreso internacional
\emph{Distant Galaxies from the Far South}
\url{https://www.astro.rug.nl/~galpatagonia/index.php} con una
presentaci\'on oral titulada \emph{Boosting the Lyman-alpha
line from stochastic IMF sampling}.

\end{document}
