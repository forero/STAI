\documentclass[12pt,spanish]{article}
\usepackage[spanish]{babel}
\selectlanguage{spanish}
\usepackage[utf8]{inputenc}
\usepackage{hyperref}
\title{{\sc Entendiendo a DESI}\\{\small\sc Reporte STAI 2017-20}}
\author{Jaime E. Forero Romero\\Departamento de F\'isica\\Universida
  de los Andes}
\begin{document}
\maketitle
\begin{abstract}
Este documento es el reporte del Semestre de Trabajo Acad\'emico
Independiente (STAI) hecho durante el per\'iodo 2017-20. 
En ese per\'iodo trabaj\'e contribuyendo al Dark Energy Spectroscopic
Instrument, un experimento de \'ultima generaci\'on dise\~nado para
estudiar la expansi\'on acelerada del Universo.
\end{abstract}

\section{Objetivos}

El objetivo principal del STAI en la propuesta que fu\'e aprobada era 
\begin{quote}
Contribuir a la colaboraci\'on internacional Dark Energy Spectroscopic
Instrument (DESI) durante una visita de cuatro meses (per\'iodo
2017-20) al Lawrence Berkeley National Laboratory en California.  
\end{quote}

Este objetivos se cumpli\'o como se ver\'a a continuaci\'on con los
resultados obtenidos.
El \'unico cambi\'o que se hizo por razones de
financiaci\'on \footnote{La financiaci\'on por parte de Berkeley no se
  aprob\'o en el monto que necesitaba y tampoco fu\'e exitosa mi
  aplicaci\'on a Fulbright. A cambio de esto tuve
  financiaci\'on de la Uni\'on Europea (\url{https://www.lacegal.com/})
  para la visita al Reino Unido y mis colegas de Corea del Sur
  tambi\'en decidieron cubrir mis gastos en la visita que les hice.}
es que la visita de cuatro meses a Berkeley Lab se dividi\'o en tres
institutos diferentes parte del Dark Energy Spectroscopic Instrument.   


Los institutos parte de DESI que se visitaron durante el STAI fueron:
\begin{itemize}
\item{Institute for Computational Cosmology de la Universidad de
  Durham en el Reino Unido.}
\item{Korea Astronomy and Space Science Institute en Corea del Sur.}
\item{Lawrence Berkeley National Laboratory en California.}
\end{itemize}

En las tres visitas estuve realizando las mismas actividades:

\begin{itemize}
\item Desarrollar y validar el m\'odulo \verb"desitarget". Este m\'odulo est\'a
  encargado de seleccionar las galaxias que ser\'an observadas por
  DESI.
  Especialmente estuve desarrollando y validando la parte del
  m\'odulo que crea esta lista de galaxias a partir de simulaciones
  cosmol\'ogicas.
\item Desarrolar y validar el m\'odulo \verb"desisim". Este m\'odulo
est\'a encargado de simular el proceso de observaciones de DESI noche
por noche. 
Especialmente estuve desarrollando y validando la parte del m\'odulo
que hace simulaciones r\'apidas de DESI. Esto permite simular en cerca
de cinco horas los resultados completos de cinco a\~nos de operaciones
de DESI.
\item Desarrollar y validar el m\'odulo
  \verb"quicksurvey_example". Este m\'odulo sirve de demostraci\'on
  sobre como simular DESI. Su utilidad es doble: 1) nuevos miembros de
  la colaboraci\'on pueden aprender a simular desi y 2) sirve de test
  de validaci\'on ante cualquier cambio de los m\'odulos de
  simulaci\'on.
\item Producir simulaciones de DESI que luego ser\'ian disponibles a
  diferentes miembros de la colaboraci\'on.
\end{itemize}

\section*{Resultados Obtenidos}

Los resultados esperados en la propuesta eran los siguientes
\begin{enumerate}
\item Diagn\'ostico de \textit{performance} de DESI a partir de los
  datos de la simulaci\'on completa del instrumento.  
\item Contribuir software para simulaci\'on de DESI en repositorios de
  acceso p\'ublico a la comunidad internacional.
\item Escribir una nota t\'ecnica interna a la colaboraci\'on DESI
  sobre los resultados principales del ejercicio de simulaci\'on.
\item Escribir y someter al menos un art\'iculo (a una revista ISI
  de primer cuartil) con resultados de este ejercicio de
  simulaci\'on que sean de inter\'es para la comunidad internacional.  
\item Solicitar una recomendaci\'on para ser   considerado como
  \texttt{Builder}  dentro de la colaboraci\'on DESI.  
 \end{enumerate}

El estado despu\'es del STAI de cada uno de los puntos anteriores es
el siguiente.

\begin{enumerate}
\item Las simulaciones se realizaron. 
  El resultado principal es que a\'un bajo las condiciones m\'as
  pesimistas de ineficiencia de instrumentos y mal clima, DESI
  lograr\'a tener el n\'umero de observaciones necesarias para
  completar sus objetivos cient\'ificos.

  Adjunto el pdf de la p\'agina
  interna de la colaboraci\'on que habla de los resultados de esta simulaci\'on:\\
  \verb"100% Quicksurvey End-to-End Simulation".
  Aqu\'i se simulan los resultados de DESI despu\'es de cinco a\~nos
  de operaciones bajo diferentes condiciones. Esto incluye: la
  selecci\'on de galaxias a observar 
  dentro de simulaciones cosmol\'ogicals, la cadencia del telescopio
  noche por noche, la asignaci\'on de fibras \'opticas a galaxias y
  finalmente la medici\'on del redshift de cada una de las galaxias
  simuladas. El resultado final son varios cat\'alogos, cada uno con
  50 millones de galaxias, como resultado final de DESI bajo
  diferentes condiciones de observaci\'on. 

  Una presentaci\'on (Data Challenge Update) con m\'as detalles se
  adjunta a este reporte. 
  
\item La contribuci\'on al software ha sido continua. Todas las
  contribuciones pueden ser vistas en el repositorio p\'ublico del
  proyecto:\\ \url{https://github.com/desihub}.

Sobresalen las contribuciones que hice en los siguientes paquetes de software:
\begin{itemize}
\item{\texttt{quicksurvey\_example}


  Soy el l\'ider en contribuciones. Este paquete demuestra la manera
  en la que se hacen las simulaciones para diagnosticar la performance
  de DESI.


  La lista detallada de contribuciones se encuentra en:


 \url{https://github.com/desihub/quicksurvey_example/graphs/contributors}}

\item \texttt{desitarget}

Soy el tercer mayor contribuidor a este paquete (de 17 desarrolladores
en total). Este paquete se encarga de crear archivos con Universos
simulados como entrada al simulador principal de DESI.

La lista detallada de contribuciones se encuentra en:

\url{https://github.com/desihub/desitarget/graphs/contributors}

\item \texttt{desisim}

Soy el quinto mayor contribuidor a este paquete (de 18 desarrolladores
en total). Este paquete es el simulador principal de DESI.

La lista detallada de contribuciones se encuentra en:

\url{https://github.com/desihub/desisim/graphs/contributors}

\end{itemize}
\item En lugar de una nota t\'ecnica se hizo una presentaci\'on en el
  DESI Collaboration Meeting en Stanford
  \url{https://kipac.stanford.edu/events/desi-collaboration-meeting}. 

  La raz\'on para no escribir la nota t\'ecnica es que se decidi\'o
  dar los resultados en forma de cat\'alogos para que cada subgrupo de
  DESI (cerca de 10 grupos diferentes) lo analizara de acuerdo a sus
  necesidades. 
  La presentaci\'on en el DESI collaboration meeting lo que hace
  entonces es resumir las caracter\'isticas de los cat\'alogos
  simulados. 
  Esta presentaci\'on (Data Challenge Update) se adjunta a este reporte.
\item Un art\'iculo (\emph{Unbiased clustering estimates with the DESI
fibre assignment}) que utiliza resultados de las simulaciones y de
  los m\'odulos desarrollados fu\'e enviado ahora (Abril 2018) y se
  adjunta a este reporte. 

\item La solicitud fue hecha y la respuesta sobre el avance hecho en
  trabajo a la colaboraci\'on es positiva. \footnote{Se adjunta la carta de
  respuesta que toma en cuenta el trabajo hecho hasta el 2016. 
  A mediados del 2018 se har\'a una nueva solicitud que
  toma en cuenta el trabajo hecho en el 2017. 
  La solicitud que incluye el trabajo en el 2017 no se puede hacer ahora debido al funcionamiento
  interno de la colaboraci\'on, solamente hasta ahora (primer
  semestre 2018) se est\'a haciendo la contabilidad de contribuciones
  hechas en el 2017. Hasta el segundo semestre del 2018 se puede tener
  una carta que incluya las contribuciones del 2017.}
\end{enumerate}


\section*{Resultados Adicionales}

Durante el semestre del STAI dos publicaciones en revistas Q1 fueron
aceptadas.

La primera publicaci\'on (\emph{Tracing the Cosmic Web}) est\'a
relacionada con el trabajo hecho durante el STAI sobre simulaciones
de estructura del Universo a gran escala. 
La segunda publicacio\'on no tiene relaci\'on directa con DESI y toma
parte de trabajo hecho con estudiantes de pregrado (\emph{Maria Camila
  Remolina-Guti\'errez}) de Uniandes


Los dos art\'iculos aparecieron publicadas a comienzos del 2018:

\begin{itemize}

\item{{\it Tracing the cosmic web}, Noam I. Libeskind,  Rien van de
  Weygaert,  Marius Cautun,  Bridget Falck, Elmo Tempel,  Tom Abel,  Mehmet
  Alpaslan,  Miguel A. Aragón-Calvo, Jaime E. Forero-Romero,  Roberto
  Gonzalez,  Stefan Gottlöber,  Oliver Hahn, Wojciech A. Hellwing , Yehuda
  Hoffman,  Bernard J. T. Jones,  Francisco Kitaura, Alexander Knebe,
  Serena Manti,  Mark Neyrinck,  Sebastián E. Nuza, Nelson Padilla,  Erwin
  Platen,  Nesar Ramachandra,  Aaron Robotham,  Enn Saar, Sergei Shandarin,
  Matthias Steinmetz,  Radu S. Stoica , Thierry Sousbie, Gustavo Yepes. 
Publicado en Monthly Notices of the Royal Astronomical Society, Volume
473, Issue 1, 1 January 2018, Pages 1195–1217,
\url{https://doi.org/10.1093/mnras/stx1976}}
\item{{\it Modelling the gas kinematics of an atypical Ly $\alpha$ emitting
    compact dwarf galaxy}, Jaime E. Forero-Romero,  Max Gronke,  Maria
  Camila Remolina-Gutiérrez, Nicolás Garavito-Camargo,  Mark Dijkstra.
  Publicado en Monthly Notices of the Royal Astronomical Society,
  Volume 474, Issue 1, 11 February 2018, Pages 12–19,
  \url{https://doi.org/10.1093/mnras/stx2699}} 

\end{itemize}

En medio del STAI el diario El Tiempo me hizo una entrevista sobre el
trabajo en DESI. La publicaci\'on se encuentra en \url{http://www.eltiempo.com/vida/ciencia/atlas-3d-mas-completo-del-universp-155450}

Al finalizar el STAI asist\'i a al congreso internacional
\emph{Distant Galaxies from the Far South}
\url{https://www.astro.rug.nl/~galpatagonia/index.php} con una
presentaci\'on oral titulada \emph{Boosting the Lyman-alpha
line from stochastic IMF sampling}.

\end{document}
