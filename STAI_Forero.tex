\documentclass[12pt,spanish]{article}
\usepackage[spanish]{babel}
\selectlanguage{spanish}
\usepackage[utf8]{inputenc}
\usepackage{hyperref}
\title{{\sc Entendiendo a DESI}\\{\small\sc Propuesta STAI 2017-20}}
\author{Jaime E. Forero Romero\\Departamento de F\'isica\\Universida
  de los Andes}
\begin{document}
\maketitle
\begin{abstract}
Este documento expone mis planes para un Semestre de Trabajo Acad\'emico
Independiente (STAI) durante el per\'iodo 2017-20. 
En ese per\'iodo planeo pasar cuatro meses en el Lawrence Berkeley
National Laboratory contribuyendo al Dark Energy Spectroscopic
Instrument, un experimento de \'ultima generaci\'on dise\~nado para
estudiar la expansi\'on acelerada del Universo.
\end{abstract}





\section*{Objetivo}

Contribuir a la colaboraci\'on internacional Dark Energy Spectroscopic
Instrument (DESI) durante una visita de cuatro meses (per\'iodo
2017-20) al Lawrence Berkeley National Laboratory en California.  


\section*{Contexto}
DESI es un experimento que har\'a un mapa 3D del universo midiendo la
distancia a 35 millones de galaxias
distantes \footnote{\url{http://desi.lbl.gov/}}. 
Este mapa ampliar\'a nuestro conocimiento sobre la
expansi\'on acelerada del Universo, hallazgo que vali\'o el premio
Nobel en  F\'isica en el 2011 a los tres astr\'onomos quienes la observaron
por primera vez. Entender esta  expansi\'on acelerada es uno de los
grandes problemas de la f\'isica fundamental del momento y uno de los
motores de la investigaci\'on en cosmolog\'ia observacional.



DESI empezar\'a su construcci\'on en el 2017 y tomar\'a datos a
partir del 1 de enero del 2019 durante un per\'iodo de 5 a\~nos. 
A la fecha, la colaboraci\'on cuenta con cerca de 300 cient\'ificos de
45 instituciones de todo el mundo. 
Los avances de DESI solamente podr\'an  ser superados con la nueva
generaci\'on de experimentos e instrumentos a mediados de la d\'ecada
del 2020. 

Uniandes hace parte formal de esta colaboraci\'on desde comienzos del
2014 gracias a mis contribuciones en el \'area de simulaciones
computacionales. 
Una contribuci\'on constante a la colaboraci\'on, como la que he
venido haciendo, permite que algunos miembros de DESI logren el
estatuto especial de \texttt{Builder}. Miembros de esta 
categor\'ia especial logran coautor\'ia en todos los art\'iculos de la
colaboraci\'on. Mi visita a Berkeley es un paso importante para
obtener la recomendaci\'on de los directores de DESI para
alcanzar ese estatuto.

\section*{Trabajo Preliminar}

Desde mi \'epoca de postdoc en UC Berkeley en el 2011-2012 (justo
antes de mi llegada a Uniades) estuve en contacto con los fundadores
de la colaboraci\'on DESI. Entre el 2012 y el 2014, ya en Uniandes,
contribu\'i con c\'odigo para simular partes de DESI y poder
guiar decisiones de dise\~no del experimento.  

Esta contribuci\'on cient\'ifica permiti\'o la entrada oficial de
Uniandes a la colaboraci\'on sin pagar los altos costos monetarios que
esto implica normalmente para otros cient\'ificos\footnote{Del orden
  de 200 mil d\'olares por investigador principal.} 

Desde el 2014 contin\'uo con mayor intensidad este trabajo de
simulaci\'on. Visito Berkeley una vez al
semestre, en promedio, para trabajar en el proyecto.
La mitad de los fondos que costean esos viajes provienen directamente
de DESI y la otra mitad de mi proyecto FAPA. 
Resultados parciales de este trabajo han sido presentados en
encuentros de la Society of Photo-Optical Instrumentation Engineers
(SPIE):
\begin{itemize}
\item 
\textit {Target allocation yields for massively multiplexed
  spectroscopic surveys with fibers},  Saunders, Will; Smedley, Scott;
Gillingham, Peter; {\bf Forero-Romero, Jaime E.}; Jouvel, Stephanie; Nord, Brian, 
Proceedings of the SPIE, Volume 9150, id. 915023 10 pp. (2014).
\end{itemize}
%
y la  American Astronomical Society  (AAS)
\begin{itemize}
\item \textit{The Dark Energy Spectroscopic Instrument (DESI): Tiling and Fiber Assignment}
Cahn, Robert N.; Bailey, Stephen J.; Dawson, Kyle S.; {\bf Forero Romero,
Jaime}; Schlegel, David J.; White, Martin;
American Astronomical Society, AAS Meeting \#225, id.336.10, (2015)
\end{itemize}



El objetivo cient\'ifico para el 2017 es simular el instrumento
completo para desarrollar estrategias que maximicen su r\'edito
cient\'ifico. 
El marco general que gu\'ia esta iniciativa de
simulaci\'on ya fue objeto de una publicaci\'on a comienzo de este a\~no:

\begin{itemize}
\item {\it SPOKES: An end-to-end simulation facility for
  spectroscopic cosmological surveys}, 
	Nord, B.; Amara, A.; R\'efr\'egier, A.; Gamper, La.; Gamper, Lu.;
        Hambrecht, B.; Chang, C.; {\bf Forero-Romero, J. E.}; Serrano, S.;
        Cunha, C.; Coles, O.; Nicola, A.; Busha, M.; Bauer, A.;
        Saunders, W.; Jouvel, S.; Kirk, D.; Wechsler, R., Astronomy
        and Computing, 15, 1, (2016)
\end{itemize}

Durante el primer semestre del 2017 haremos el trabajo de
simulaci\'on. En esto esperamos utilizar 10 millones de horas de CPU
para generar 200 TB de datos.
Durante mi visita a Berkeley en el segundo semestre del 2017, vamos
a dedicarnos al an\'alisis detallado de estos datos para diagnosticar
la \emph{performace} esperada del experimento y sugerir estrategias para
maximizar el r\'edito cient\'ifico de DESI.  


\section*{Resultados Esperados}
\begin{itemize}
\item Diagn\'ostico de \textit{performance} de DESI a partir de los
  datos de la simulaci\'on completa del instrumento.
\item Contribuir software para simulaci\'on de DESI en repositorios de
  acceso p\'ublico a la comunidad internacional.
\item Escribir una nota t\'ecnica interna a la colaboraci\'on DESI
  sobre los resultados principales del ejercicio de simulaci\'on.
\item Escribir y someter al menos un art\'iculo (a una revista ISI
  de primer cuartil) con resultados de este ejercicio de
  simulaci\'on que sean de inter\'es para la comunidad internacional.  
\item Solicitar una recomendaci\'on para ser   considerado como
  \texttt{Builder}  dentro de la colaboraci\'on DESI.  
 \end{itemize}

\section*{Recursos Necesarios}

Recursos para viajes y publicaciones vienen de mi proyecto
\texttt{Simulaciones y Observaciones del Universo a Gran Escala} (SOUGE). Este
proyecto de 36 meses de duraci\'on es financiado por COLCIENCIAS. Su
ejecuci\'on debe empezar en Octubre del 2016. 

El Lawrence Berkeley National Laboratory va proveerme de espacio de
oficina y acceso a hardware/software de la colaboraci\'on DESI. Esta
es una continuaci\'on natural de lo que ya ha venido pasando con mis
visitas desde el 2014. 

Tambi\'en envi\'e una solicitud para obtener una Fulbright
Scholarship. Esto me permitir\'ia tener fondos adicionales, pero en
caso de ser denegada no impedir\'ia realizar el STAI con \'exito.

\section*{Cronograma}
\begin{itemize}
\item Agosto 2017. An\'alisis de los datos de la simulaci\'on completa
  de DESI. 
\item Septiembre 2017. Redacci\'on del reporte interno para la
  colaboraci\'on sobre los resultados de la simulaci\'on. 
Incluye
  sugerencias para maximizar la \emph{performance} del instrumento. 
\item Octubre 2017. Redacci\'on del art\'iculo internacional
  presentando los resultados m\'as interesantes del ejercicio de
  simulaci\'on. 
\item Noviembre 2017. Presentaci\'on de los resultados en encuentros
  de la colaboraci\'on DESI. Solicitud de la recomendaci\'on como
  \texttt{\bf Builder}.
\end{itemize}

\end{document}
